\section{Introduzione}

\subsection{ArtemisCartesio}

ArtemisCartesio rappresenta un'iniziativa pionieristica nel campo dell'esplorazione spaziale, sviluppata da un'agenzia spaziale internazionale per supportare missioni lunari avanzate. L'obiettivo principale è sfruttare tecnologie all'avanguardia per raccogliere, analizzare e gestire i dati provenienti da sensori, robot autonomi e membri dell'equipaggio impegnati in operazioni sulla superficie lunare. Questa piattaforma è progettata per affrontare le complessità logistiche e tecniche di missioni scientifiche, garantendo al contempo efficienza, sicurezza e affidabilità.\\

\noindent
Il nome \textbf{Artemis Cartesio} unisce il simbolismo della dea Artemide, legata alla luna e all’esplorazione, con il razionalismo scientifico di Cartesio, rappresentando così l’equilibrio tra avventura e metodo. Inoltre, le iniziali \textbf{A.C.} omaggiano i creatori del progetto, conferendo un tocco personale al nome.

\subsection{Specifiche del Sistema}

Il progetto prevede lo sviluppo di una piattaforma informatica che integri:
\begin{itemize}
    \item \textbf{Gestione delle missioni}: Archiviazione di dettagli quali obiettivo, stato, data di inizio e fine, e gestione delle risorse coinvolte.
    \item \textbf{Monitoraggio dei sensori}: Registrazione delle informazioni relative ai sensori (coordinate, tipo, stato operativo) e delle rilevazioni effettuate.
    \item \textbf{Gestione delle anomalie}: Identificazione e registrazione di anomalie, con dati relativi a data, ora, livello di priorità e causa.
    \item \textbf{Interventi e manutenzione}: Programmazione e tracciamento degli interventi per risolvere anomalie, con dettagli quali esito e descrizione delle operazioni effettuate.
    \item \textbf{Reportistica e analisi statistiche}: Compilazione di report da parte dei membri dell'equipaggio e analisi statistiche per ottimizzare le operazioni.
\end{itemize}

\subsection{Struttura del Progetto}

Per implementare il sistema, il progetto include le seguenti fasi principali:
\begin{itemize}
    \item \textbf{Progettazione della base di dati}:
    \begin{itemize}
        \item \textit{Concettuale}: Definizione delle entità, delle relazioni e degli attributi principali.
        \item \textit{Logica}: Traduzione dello schema concettuale in un modello relazionale.
        \item \textit{Fisica}: Ottimizzazione dello schema logico per l'implementazione in Oracle DBMS.
    \end{itemize}
    \item \textbf{Ottimizzazione delle prestazioni}:
    \begin{itemize}
        \item Creazione di indici per velocizzare le operazioni.
        \item Progettazione di strategie di backup, recovery e replicazione per garantire l'affidabilità.
    \end{itemize}
    \item \textbf{Implementazione SQL}:
    \begin{itemize}
        \item Creazione di stored procedure, trigger, query e viste per gestire le operazioni e supportare l'automazione.
    \end{itemize}
    \item \textbf{Interfaccia web-based}:
    \begin{itemize}
        \item Sviluppo di un'interfaccia utente tramite Oracle APEX per semplificare l'interazione con il sistema.
    \end{itemize}
\end{itemize}

\subsection{Obiettivi}

Il sistema è progettato per:
\begin{itemize}
    \item Migliorare la gestione e il coordinamento delle missioni lunari.
    \item Consentire un monitoraggio avanzato e in tempo reale delle operazioni.
    \item Supportare l'analisi statistica dei dati raccolti per prendere decisioni informate.
\end{itemize}

\subsection*{Conclusione}

Questo progetto non solo rappresenta un passo avanti nella gestione delle missioni spaziali, ma si pone come esempio di integrazione tra tecnologie avanzate e necessità operative. L'approccio metodologico adottato garantisce una base solida per affrontare sfide future nell'esplorazione lunare e spaziale.
